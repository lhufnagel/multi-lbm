%%%%%%%%%%%%%%%%%%%%%%%%%%%%%%%%%%%%%%%%%
% Plain Cover Letter
% LaTeX Template
% Version 1.0 (28/5/13)
%
% This template has been downloaded from:
% http://www.LaTeXTemplates.com
%
% Original author:
% Rensselaer Polytechnic Institute 
% http://www.rpi.edu/dept/arc/training/latex/resumes/
%
% License:
% CC BY-NC-SA 3.0 (http://creativecommons.org/licenses/by-nc-sa/3.0/)
%
%%%%%%%%%%%%%%%%%%%%%%%%%%%%%%%%%%%%%%%%%

%----------------------------------------------------------------------------------------
%	PACKAGES AND OTHER DOCUMENT CONFIGURATIONS
%----------------------------------------------------------------------------------------

\documentclass[11pt]{letter} % Default font size of the document, change to 10pt to fit more text

\usepackage{newcent} % Default font is the New Century Schoolbook PostScript font 
%\usepackage{helvet} % Uncomment this (while commenting the above line) to use the Helvetica font

% Margins
\topmargin=-1in % Moves the top of the document 1 inch above the default
\textheight=8.5in % Total height of the text on the page before text goes on to the next page, this can be increased in a longer letter
\oddsidemargin=-10pt % Position of the left margin, can be negative or positive if you want more or less room
\textwidth=6.5in % Total width of the text, increase this if the left margin was decreased and vice-versa

\let\raggedleft\raggedright % Pushes the date (at the top) to the left, comment this line to have the date on the right

\begin{document}

%----------------------------------------------------------------------------------------
%	ADDRESSEE SECTION
%----------------------------------------------------------------------------------------

\begin{letter}{SIAM \\
3600 Market Street, 6th Floor \\
Philadelphia, PA 19104-2688 USA}

%----------------------------------------------------------------------------------------
%	YOUR NAME & ADDRESS SECTION
%----------------------------------------------------------------------------------------

\begin{center}
\large\bf Lorenz Hufnagel, Daniel Zint \\ % Your name
%\vspace{20pt} \hrule height 1pt % If you would like a horizontal line separating the name from the address, uncomment the line to the left of this text
Chair for System-Simulation \\ 
Friedrich Alexander Universit\"at Erlangen-N\"urnberg \\ 
91058 Erlangen, Germany \\
lorenz.hufnagel@studium.uni-erlangen.de, daniel.zint@fau.de
\end{center} 
\vfill

\signature{Lorenz Hufnagel, Daniel Zint} % Your name for the signature at the bottom

%----------------------------------------------------------------------------------------
%	LETTER CONTENT SECTION
%----------------------------------------------------------------------------------------

\opening{Dear Sir or Madam,} 

% Der Stil und der Inhalt richten sich nach dem was ich auf Google zu dem Thema gefunden habe. Ob dies tatsächlich so richtig ist, sollten wir aber nochmal absegnen lassen.

% The first paragraph introduces the writer, and follows the basic format of the intro para of the job cover letter described in this post (ie, field, Ph.D. institution and year, current institutional affiliation and status, and general focus of work). It then states that the writer is submitting a manuscript for review.

We are students of the Bavarian Graduate School of Computational Engineering. We are both students at the Friedrich Alexander Universit\"at Erlangen-N\"urnberg and participate in the program Computational Engineering. Whereas L. Hufnagel also did his Bachelors in Computational Engineering, D. Zint did Mechatronics in his Bachelors.

% The second paragraph covers the topic of the manuscript. This will be a crisp 4-5 sentences that will give a title and describe the topic, the specific material/data covered, the theoretical orientation or approach, any special issues of methodology if important, and, most importantly, the core argument.

One part of our program of study is a project. Our task was to reimplement the algorithm described in the paper "A lattice Boltzmann method for immiscible multiphase flow simulations using the level set method" of G. Th\"ommes. Furthermore, we did some parameter studies on the algorithm. The method described in the paper uses the lattice Boltzmann method to simulate the flow field of the fluids and the level set method to compute the movement of the interface between the different fluids. It couples the two methods by creating a new boundary condition for the interface in the lattice Boltzmann method. We took a closer look at this coupling and examined possibilities to optimize the code by changing parameters that influence the interacting behavior of the two methods.

% The third paragraph will be shorter, and will take about 2-3 sentences to describe the manuscript’s contribution to the field and the suitability of the manuscript to this particular journal based on topic, theme, or methodological or theoretical approach, with reference to other work recently published in the journal.

With this paper we will give some advices on what is to mind if someone wants to reimplement the method of G. Th\"ommes. Researcher can use our parameter studies to find their own sweet spot between performance and accuracy.

% The final paragraph will list 2-3 possible reviewers for the manuscript, and will thank the editor for considering the manuscript for publication. Contact info can be added here.

Possible reviewers for this paper would be G. Th\"ommes, J. Becker, and M. Junk. Thank you for considering this paper for puplication!

\closing{Sincerely yours,}


%\encl{Curriculum vitae, employment form} % List your enclosed documents here, comment this out to get rid of the "encl:"

%----------------------------------------------------------------------------------------

\end{letter}

\end{document}